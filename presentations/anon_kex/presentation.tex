% Set the document class and theme
\documentclass{beamer}
\usetheme{CambridgeUS}
\setbeamertemplate{caption}[numbered]
\setbeamertemplate{theorems}[numbered]
\setbeamerfont{footnote}{size=\tiny}

\usepackage{./presentation_macros}

\addbibresource{references.bib}

% Add presentation data here

% Text in the square brackets `[]' are shown in the footer. If not mentioned,
% then text in the curly braces `{}' are used as theme defaults.

\title[AKA for V2X]{Anonymous Key Agreements for V2X Communication}
\date{May 1, 2024}
\author{Gautam Singh}
\institute[IITH]{Indian Institute of Technology Hyderabad}

% Presentation begins here

\begin{document}
    \maketitle
    \tableofcontents
    \section{Introduction}
    
    \begin{frame}
        \frametitle{V2X Related Terminology}
        \begin{figure}
            \centering
            \resizebox{.8\textwidth}{!}{\begin{tikzpicture}[
    mindmap,
    concept color=red!55!black,
    text=white
]
    \node [concept] {\textbf{V2X}\\\small{Vehicle-to-Everything}}
    child[grow=150] {node[concept] {\textbf{V2D}\\\tiny{Vehicle-to-Device}}}
    child[grow=210] {node[concept] {\textbf{V2G}\\\tiny{Vehicle-to-Grid}}}
    child[grow=0] {
        node[concept] {\textbf{V2N}\\\tiny{Vehicle-to-Network}}
        child[grow=60] {node[concept] {\textbf{V2C}\\\tiny{Vehicle-to-Cloud}}}
        child[grow=120] {node[concept] {\textbf{V2P}\\\tiny{Vehicle-to-Pedestrian}}}
        child[concept color=green,grow=240,text=black] {node[concept] {\textbf{V2V}\\\tiny{Vehicle-to-Vehicle}}}
        child[concept color=green,grow=300,text=black] {node[concept] {\textbf{V2I}\\\tiny{Vehicle-to-Infrastructure}}}
    };
\end{tikzpicture}}
            \caption{A breakdown of V2X.}
        \end{figure}
    \end{frame}

    \begin{frame}
        \frametitle{Message Types in V2X}
        \begin{enumerate}
            \item<1-> \textbf{Cooperative Awareness Messages} (CAMs)
            \footcite{etsi-en-302-637} and \textbf{Basic Safety Messages} (BSMs)
            \footcite{J2735_202309V2XCommunications}.
            \begin{enumerate}
                \item Exchanged between vehicles to create awareness and support
                cooperative performance of vehicles in the road network.
                \item Includes status information such as time, position, speed,
                active systems, vehicle dimensions, etc.
                \item Broadcasted unencrypted in 5.9 GHz channel (ETSI ITS-G5).
                \item \textbf{Huge privacy concerns and threats!}
            \end{enumerate}
            \item<2-> Other types of messages
            \begin{enumerate}
                \item \textbf{Signal Phase and Timing} (SPaT)
                \item \textbf{Roadside Infrastructure Information} (MAP)
            \end{enumerate} 
        \end{enumerate}
    \end{frame}

    \begin{frame}
        \frametitle{Motivation and Goals}
        \begin{enumerate}
            \item<1-> Do we \emph{really} need to encrypt CAMs?
            \begin{itemize}
                \item Google (Maps) may already be profiling us!
                \item Focus on encrypting more sensitive messages and
                information sent less frequently.
            \end{itemize}
            \item<2-> Unlimited privacy.
            \item<3-> Negligible storage and bandwidth overheads.
            \item<4-> Better security guarantees (privacy, authenticity,
            confidentiality).
        \end{enumerate}
    \end{frame}

    \section{Preliminaries}

    \begin{frame}
        \frametitle{Pairings}
        \begin{definition}{Pairing}
            Let \(\bG_0 = \abrak{g_0}\), \(\bG_1 = \abrak{g_1}\), \(\bG_T\) be
            three cyclic groups of prime order \(q\). A \emph{pairing} is an
            efficiently computable function \(e: \bG_0 \times \bG_1 \rightarrow
            \bG_T\) satisfying the following properties:
            \begin{enumerate}
                \item \emph{bilinear}: for all \(u, u^{\prime} \in \bG_0\) and
                \(v, v^{\prime} \in \bG_1\), we have
                \begin{align}
                    e\brak{uu^{\prime}, v} &= e\brak{u,v}e\brak{u^{\prime},v} \\
                    e\brak{u, vv^{\prime}} &= e\brak{u,v}e\brak{u, v^{\prime}}
                \end{align}
                \item \emph{non-degenerate}: \(g_T := e\brak{g_0, g_1}\) is a
                generator of \(\bG_T\).
            \end{enumerate}
        \end{definition}
        \begin{enumerate}
            \item Here, \(\bG_0\) and \(\bG_1\) are called \emph{source
            groups} and \(\bG_T\) is called the \emph{target group}.
            \item When \(\bG_0 = \bG_1\), the pairing is said to be
            \emph{symmetric}.
        \end{enumerate}
    \end{frame}

    \begin{frame}
        \frametitle{Anonymous Key Agreement}
        \begin{enumerate}
            \item<1-> A key agreement protocol where two parties agree on a
            shared secret key, without being able to determine the other party.
            \item<2-> Pairing-based anonymous key agreement for V2X
            \begin{itemize}
                \item Clients should authenticate each other.
                \item Clients should not be able to determine the identity of
                each other.
            \end{itemize}
            \item<3-> We use a pairing-based anonymous key agreement involving a
            private key generator (PKG).
            \begin{enumerate}
                \item PKG has its own master private and public key.
                \item PKG uses master secret key to generate secret keys for
                clients.
                \item Clients use this secret key to establish the shared secret
                key.
            \end{enumerate}
        \end{enumerate}
    \end{frame}

    \begin{frame}
        \frametitle{Attributes, Credentials, Anonymous Credentials}
        \begin{enumerate}
            \item<1-> \textbf{Attributes}: Labels associated with a user that
            describe them fully, such as role of a user.
            \item<2-> \textbf{Credential}: Data possessed by a user that
            demonstrates their attributes.
            \item<3-> \textbf{Anonymous Credential}: Data possessed by a user
            that demonstrates their attributes, \emph{without revealing any
            additional information} about their identity.
            \item<4-> For V2X, we require anonymous credentials to be issued to
            vehicles regularly to ensure anonymity as well as to check
            legitimacy of that vehicle.
            \item<5-> We use DGSA (Dynamic Group Signatures with Attributes), an
            anonymous credential signature scheme using attributes. The
            anonymous credential can be abstracted as a \textbf{randomizable}
            group element which proves legitimacy of user.
        \end{enumerate}
    \end{frame}

    \section{Our Proposition}
    \begin{frame}
        \frametitle{Proposed Message Flow Diagram}
        \begin{figure}[!ht]
            \centering
            \includegraphics[width=\columnwidth]{figs/flow.png}
            \caption{Message flow of the proposed scheme.}
            \label{fig:new-flow}
        \end{figure}
    \end{frame}

    \begin{frame}
        \frametitle{Proposed Message Flow}
        \begin{enumerate}
            \item<1-> Enrollment authority issues certificate to vehicle.
            \item<2-> Issuer issues DGSA credential and vehicle secret key after
            verifying certificate.
            \begin{itemize}
                \item This secret key is different from secret key associated
                with certificate.
                \item DGSA credentials guarantee authenticity.
                \item Anonymous key agreement ensures that user identities
                remain anonymous throughout communication.
                \item This is done periodically every \emph{epoch}.
            \end{itemize}
            \item<3-> Vehicles exchange DGSA-signed randomized psuedonyms to
            generate shared key for futher communicaton.
        \end{enumerate}
    \end{frame}

    \begin{frame}
        \frametitle{Analysis}
        \begin{enumerate}
            \item<1-> \textbf{Advantages}
            \begin{itemize}
                \item Fully anonymous communication, unlimited privacy.
                \item Others cannot identify who is communicating.
            \end{itemize}
            \item<2-> \textbf{Disadvantages}
            \begin{itemize}
                \item A lot of pairing computations, for DGSA and for anonymous
                key agreement. Incurs computational overheads.
                \item Works for single-hop connections only.
            \end{itemize}
        \end{enumerate}
    \end{frame}

    \section{Conclusion}
    \begin{frame}
        \frametitle{Future Work}
        \begin{enumerate}
            \item Encrypt V2X messages like CAMs.
            \item Improve efficiency of the present work.
            \item A new workflow for encryption using zones and zone managers.
        \end{enumerate}
    \end{frame}

\end{document}
