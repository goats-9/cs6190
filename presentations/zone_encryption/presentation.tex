% Set the document class and theme
\documentclass{beamer}
\usetheme{CambridgeUS}
\setbeamertemplate{caption}[numbered]
\setbeamertemplate{theorems}[numbered]

\usepackage{./presentation_macros}

% Add presentation data here

% Text in the square brackets `[]' are shown in the footer. If not mentioned,
% then text in the curly braces `{}' are used as theme defaults.

\title[ZE-V2V]{Zone Encryption with Anonymous Authentication for V2V Communication \\
\small Jan Camenisch, Manu Drijvers, Anja Lehmann, Gregory Neven and Patrick Towa}
\date{April 23, 2024}
\author{Gautam Singh}
\institute[]{Indian Institute of Technology Hyderabad}

% Presentation begins here

\begin{document}
    \maketitle
    \tableofcontents
    \section{Introduction}
    
    \begin{frame}
        \frametitle{V2X Related Terminology}
        \begin{figure}
            \centering
            \resizebox{.8\textwidth}{!}{\begin{tikzpicture}[
    mindmap,
    concept color=red!55!black,
    text=white
]
    \node [concept] {\textbf{V2X}\\\small{Vehicle-to-Everything}}
    child[grow=150] {node[concept] {\textbf{V2D}\\\tiny{Vehicle-to-Device}}}
    child[grow=210] {node[concept] {\textbf{V2G}\\\tiny{Vehicle-to-Grid}}}
    child[grow=0] {
        node[concept] {\textbf{V2N}\\\tiny{Vehicle-to-Network}}
        child[grow=60] {node[concept] {\textbf{V2C}\\\tiny{Vehicle-to-Cloud}}}
        child[grow=120] {node[concept] {\textbf{V2P}\\\tiny{Vehicle-to-Pedestrian}}}
        child[concept color=green,grow=240,text=black] {node[concept] {\textbf{V2V}\\\tiny{Vehicle-to-Vehicle}}}
        child[concept color=green,grow=300,text=black] {node[concept] {\textbf{V2I}\\\tiny{Vehicle-to-Infrastructure}}}
    };
\end{tikzpicture}}
            \caption{A breakdown of V2X.}
        \end{figure}
    \end{frame}

    \begin{frame}
        \frametitle{Message Types in V2X}
        \begin{enumerate}
            \item \textbf{Cooperative Awareness Messages} (CAMs)
            \cite{etsi-en-302-637} and \textbf{Basic Safety Messages} (BSMs)
            \begin{enumerate}
                \item Exchanged between vehicles to create awareness and support
                cooperative performance of vehicles in the road network.
                \item Includes status information such as time, position, speed,
                active systems, vehicle dimensions, etc.
            \end{enumerate}
            \pause
            \item Other types of messages
            \begin{enumerate}
                \item \textbf{Signal Phase and Timing} (SPaT)
                \item \textbf{Roadside Infrastructure Information} (MAP)
            \end{enumerate} 
        \end{enumerate}
    \end{frame}

    \begin{frame}
        \frametitle{V2X and Cryptology}
        \begin{enumerate}
            \item CAMs broadcasted unencrypted in 5.9 GHz channel (ETSI ITS-G5).
            \begin{enumerate}
                \item Frequently broadcast: 1 CAM per second in US, 10 per
                second in EU.
                \item Easy to intercept.
                \item Leak sensitive information about the vehicle owners.
                \pause
                \item \textbf{Huge privacy concerns and threats!}
            \end{enumerate}
            \pause
            \item Encryption impractical, since CAMs \emph{must} be decrypted by
            nearby vehicles in a highly dynamic environment.
            \begin{enumerate}
                \item But CAMs \emph{have to} be encrypted because of the data
                they carry!
            \end{enumerate}
            \pause
            \item Instead, focus on \emph{privacy-preserving authentication}.
            \begin{enumerate}
                \item Ensuring a message is issued by a ``genuine'' vehicle.
                \item ``Genuine'' vehicles must be untraceable.
            \end{enumerate}
        \end{enumerate}
    \end{frame}

    \begin{frame}
        \frametitle{V2X and Cryptology}
        \begin{enumerate}
            \item Deployed systems
            \begin{enumerate}
                \item Use short-term \textbf{pseudonym certificates} (100 per
                week in EU, 20 per week in US), rotate between them.
                \item Trade-off between security (Sybil resistance), privacy and
                efficiency (storage and bandwidth costs).
            \end{enumerate}
            \pause
            \item Proposed systems
            \begin{enumerate}
                \item Stronger privacy and security guarantees.
                \item Do not fit the \emph{stringent bandwidth constraint} of
                \textbf{300 bytes per CAM}, thus they are impractical.
            \end{enumerate}
        \end{enumerate}
    \end{frame}

    \begin{frame}
        \frametitle{Motivation and Goals}
        \begin{enumerate}
            \item Aim to tackle the problem of privacy.
            \item Address the problem of authenticity and confidentiality in
            combination \emph{for the first time} (important to mention this?).
            \item Meet (bandwidth) requirements.
            \item Efficient encryption scheme (symmetric-key crypto).
            \item Better security guarantees (privacy, authenticity,
            confidentiality).
        \end{enumerate}
    \end{frame}

    \section{Preliminaries}
    \begin{frame}
        \frametitle{Preliminaries}
        This is a slide with the list of preliminaries needed to understand ZE.
        We pick up the ones not covered (top down approach, start from ZE and 
        then explain these if needed). Must list purpose of each preliminary 
        here.
        \begin{enumerate}
            \item Pairing Groups
            \item Hardness Assumptions (lot of notation, may be hard to grasp)
            \begin{enumerate}
                \item SDL
                \item q-MSDH-1
            \end{enumerate}
            \item Deterministic Authenticated Encryption (how much to cover?)
            \item PS Signatures
            \item DGS+A
        \end{enumerate}
    \end{frame}

    \section{Zone Encryption}
    
    \begin{frame}
        \frametitle{Syntax of ZE Scheme}
        \begin{enumerate}
            \item Define zone, payload, epoch.
            \item Explain all the algos used.
        \end{enumerate}
    \end{frame}

    \begin{frame}
        \frametitle{Security Properties}
        Attack game and definitions 3-6 (are theorems 4-8 needed?)
    \end{frame}

    \begin{frame}
        \frametitle{Instantiation of ZE and Efficiency}    
        (Is it worth mentioning section 4.4.1 or can we leave this?)
    \end{frame}

    \begin{frame}
        \frametitle{Summary of ZE}
        Table 2 of the paper.
    \end{frame}

    \section{Group Signatures with Attributes}
    \begin{frame}
        \frametitle{DGS+A}
        Sub-headings
        \begin{enumerate}
            \item Syntax
            \item Security properties (no proofs)
            \item Instantiation from PS
            \item Can be extended to threshold opening (should be a slide or
            only a mention during talk?)
        \end{enumerate}
    \end{frame}

    \section{Conclusion}
    \begin{frame}
        \frametitle{Challenges in Deploying ZE}
        Section 4.6
    \end{frame}

    \begin{frame}
        \frametitle{Future Improvements}
        Section 4.6, brief and top-level idea of mini-project if time permits.
    \end{frame}

    \section{References}
    \begin{frame}[allowframebreaks]
        \frametitle{References}
        \bibliography{references.bib}
    \end{frame}
\end{document}
