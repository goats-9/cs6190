% Set the document class and theme
\documentclass{beamer}
\usetheme{CambridgeUS}
\setbeamertemplate{caption}[numbered]
\setbeamertemplate{theorems}[numbered]

\usepackage{./presentation_macros}

% Add presentation data here

% Text in the square brackets `[]' are shown in the footer. If not mentioned,
% then text in the curly braces `{}' are used as theme defaults.

\title[ZE-V2V]{Zone Encryption with Anonymous Authentication for V2V Communication \\
\small Jan Camenisch, Manu Drijvers, Anja Lehmann, Gregory Neven and Patrick Towa}
\date{April 23, 2024}
\author{Gautam Singh}
\institute[]{Indian Institute of Technology Hyderabad}

% Presentation begins here

\begin{document}
    \maketitle
    \tableofcontents
    \section{Introduction}
    
    \begin{frame}
        \frametitle{What is V2X?}
        \begin{enumerate}
            \item Introduce V2X-related terms like pseudonym, RSU, CAM, etc.
            \item Mention US/Europe standards for V2X.
        \end{enumerate}
    \end{frame}

    \begin{frame}
        \frametitle{V2X and Cryptology}
        \begin{enumerate}
            \item 100/20 pseudonyms per week
            \item 300 byte per CAM bandwidth constraint
            \item Crypto implications of the above
        \end{enumerate}
    \end{frame}

    \begin{frame}
        \frametitle{Motivation and Goals}
        \begin{enumerate}
            \item Aim to tackle the problem of privacy.
            \item Address the problem of authenticity and confidentiality in
            combination \emph{for the first time} (important to mention this?).
            \item Meet (bandwidth) requirements.
            \item Efficient encryption scheme (symmetric-key crypto).
            \item Better security guarantees (privacy, authenticity,
            confidentiality).
        \end{enumerate}
    \end{frame}

    \section{Preliminaries}
    \begin{frame}
        \frametitle{Preliminaries}
        This is a slide with the list of preliminaries needed to understand ZE.
        We pick up the ones not covered.
        \begin{enumerate}
            \item Pairing Groups
            \item Hardness Assumptions (lot of notation, may be hard to grasp)
            \begin{enumerate}
                \item SDL
                \item q-MSDH-1
            \end{enumerate}
            \item Deterministic Authenticated Encryption (how much to cover?)
            \item PS Signatures
        \end{enumerate}
    \end{frame}

    \section{Group Signatures with Attributes}
    \begin{frame}
        \frametitle{DGS+A}
        Sub-headings
        \begin{enumerate}
            \item Syntax
            \item Security properties (no proofs)
            \item Instantiation from PS
            \item Can be extended to threshold opening (should be a slide or
            only a mention during talk?)
        \end{enumerate}
    \end{frame}

    \section{Zone Encryption}
    
    \begin{frame}
        \frametitle{Syntax of ZE Scheme}
        \begin{enumerate}
            \item Define zone, payload, epoch.
            \item Explain all the algos used.
        \end{enumerate}
    \end{frame}

    \begin{frame}
        \frametitle{Security Properties}
        Attack game and definitions 3-6 (are theorems 4-8 needed?)
    \end{frame}

    \begin{frame}
        \frametitle{Instantiation of ZE and Efficiency}    
        (Is it worth mentioning section 4.4.1 or can we leave this?)
    \end{frame}

    \begin{frame}
        \frametitle{Summary of ZE}
        Table 2 of the paper.
    \end{frame}

    \section{Conclusion}
    \begin{frame}
        \frametitle{Challenges in Deploying ZE}
        Section 4.6
    \end{frame}

    \begin{frame}
        \frametitle{Future Improvements}
        Section 4.6, brief and top-level idea of mini-project if time permits.
    \end{frame}

    % \section{References}
    % \begin{frame}[allowframebreaks]
    %     % \frametitle{References}
    %     \bibliography{references.bib}
    % \end{frame}
\end{document}
